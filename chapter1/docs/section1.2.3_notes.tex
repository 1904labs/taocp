\documentclass[12pt]{article}
\title{Sums and Products}
\begin{document}
\maketitle
% \begin{slide}
  \begin{center}
    Sums
  \end{center}  
   Sum of $a_1,a_2 \ldots a_n$ is written as $$\sum_{j=1}^n {a_j} $$ or $$ \sum_{1 \le j \le n}a_j $$

   $j$ is {\sl dummy} or {\sl index} variable introduced just for the purpose of notation.

   Context is important - {\sl e.g.} $$ \sum _ {j \le k}{j+k \choose  2j -k}$$

   In the above case only if either $i$ or $j$ (not both) have exterior significance.

   Finite/Infinte summation -

   Example of finite summation ---

   $$ \sum_{1 \le j <10}{a_j} $$

   Example of infinite summation ---

   $$ \sum_{j \ge 1}{a_j} =  \sum_{j=1}^{\infty}{a_j} $$
   
   Divergent and convergent summation -

   $$  \sum_{R(j)}a_j = \left( \lim_{n\to\infty}\sum_{\scriptstyle R(j) \atop \scriptstyle 0  \le j < n} a_j \right) +
   \left( \lim_{n\to\infty}\sum_{\scriptstyle R(j) \atop  \scriptstyle -n \le j < 0}  a_j \right)  $$ 

   If the above limits exits the summation is called {\sl convergent} otherwise its a {\sl divergent} summation.

   Four algebraic operations on summations ---

   \begin{enumerate}
     \begin{item}
       {\sl The distributive law}, for product of sums
       $$ \left(\sum_{R(i)}a_i\right) \left(\sum_{S(j)}b_j\right) = \sum_{R(i)}\left(\sum_{S(j)}a_ib_j\right) = \sum_{R(i)}\sum_{S(j)}a_{ij} $$
     \end{item}
     \begin{item}
       {\sl The Change of variable}
       $$\sum_{R(i)}a_i = \sum_{R(j)}a_j = \sum_{R(p(j))}a_{p(j)} $$

       $$\sum_{1 \le j \le n}a_j = \sum_{1 \le j-1 \le n }a_{j-1} = \sum_{2 \le j \le n+1}a_{j-1} $$
       
     \end{item}
     \begin{item}
       {\sl Interchanging order of summation}
       $$\sum_{R(i)}\sum_{S(j)} a_{ij} = \sum_{S(j)}\sum_{R(i)} a_{ij} $$

       $$\sum_{i=1}^n\sum_{j=1}^i a_{ij} = \sum_{j=1}^n\sum_{i=j}^n a_{ij} $$
     \end{item}
     \begin{item}
       {\sl Manipulating the domain}

       $$ \sum_{R(j)} a_j +  \sum_{S(j)} a_j  =  \sum_{R(j) or S(j)} a_j +  \sum_{R(j) and S(j)} a_j $$
     \end{item}
   \end{enumerate}
   
   Bracket Notation and {\sl Kronecker delta}

   $$[\mbox{statement}] = \Biggr\{\begin{array}{ll} 1 & \mbox{if the statement is true}\\
     0 & \mbox{if the statement is false.}
     \end{array}
   $$

   $$\delta_{ij} = [i = j] = \Biggr\{ \begin{array}{ll} 1 & \mbox{if $i = j$} \\
     0 &  \mbox{if $i \ne j$} 
   \end{array} $$

%\end{slide}

%\begin{slide}
  \begin{center}
    Products
  \end{center}

  Default value of Product is 1 not 0

  \begin{equation}
    \prod_{R(j)}a_j
  \end{equation}

  {\large \bf Notes from Concrete Math}

  {\bf Delimited form of summation}
  \begin{equation}
    \sum_{k=1}^{n} a_k 
  \end{equation}

  (2) is delemited form of summation.
  In Above each $a_k$ is a term.
  Incidentally the quantity after $\sum$ (here $a_k$) is called {\it summand}.
  The index variable $k$ is said to be {\it bound} to $\sum$ sign because $k$ in $a_k$ is unrelated to appearances of $k$ outside Sigma-notation.

  {\bf Generalized Form of summation}
  It turns of that a generazlized form of summation is even more useful than the delimited form: We simply write one or more conditions under the $\sum$ to specify the set of indices over which summation should take place.Example

  \begin{equation}
    \sum_{\scriptstyle 1 \le k \le n \atop \scriptstyle {k odd}} a_k
  \end{equation}

  Efficiency of computation is not same as efficiency of understanding.
\end{document}
