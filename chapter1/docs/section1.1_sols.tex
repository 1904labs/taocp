\documentclass[12pt]{article}
\title{Solutions to Exercises from The Art of Computer Programming, by Donald Knuth}
% \author{Amit Jha, Collin Bell}
% \date{December 5,2020}

\begin{document}

\maketitle
\section{Chapter 1}
\subsection{Solutions for section1.1} 
\begin{enumerate}
\begin{item}
Using a temp variable $t$ the values can be rotated like this -

$ t \leftarrow a, a \leftarrow b, b \leftarrow c, c \leftarrow d, d \leftarrow t $ 
\end{item}


\begin{item}
  At step {\bf[E3]} $r$ is assigned to $n$ and $n$ to $m$. As $r$ is reminder of division of $m$ by $n$, $r$ should be $< n$. Hence $m < n$. 
\end{item}

\begin{item}

  Below are the steps of modified algorithm {\bf[F]} which takes $m$ and $n$ as input.
  
  {\bf[F1]} Divide $m$ by $n$ and let the reminder by $r$.

  {\bf[F2]} if $r = 0$ return $n$. Terminate

  {\bf[F3]} Invoke {\bf[F]} with $n$, $r$ as input and return result.
\end{item}


\begin{item}

  57
  
\end{item}


\begin{item}

  From the procedure reading the book following properties are missing which means it's not a proper algorithm.
  \begin{itemize}
    \begin{item}
      Finiteness is missing - the whole procedure goes in a loop and does not actually terminate.
    \end{item}

    \begin{item}
      Output is missing - the procedure does not a definite output.
    \end{item}

    \begin{item}
      Effectiveness is missing - The steps cannot be done on pencil/paper or a real computer realistically.
    \end {item}

    Comparison with {\bf[E]}:
    {\bf[E]} terminates after finite number of steps and is effective(steps can be performed on pen and paper and has definite output.
  \end{itemize}
\end{item}

\begin{item}

 The answer should be close to 3. I dint calculate the exact value though.
  
\end{item}

\begin{item}

  $T_{m} +1 = U_{m}$

\end{item}


\end{enumerate}

\subsection{Mathematics Preliminaries}
\subsubsection{Solutions for section 1.2.1}
\begin{enumerate}
  \begin{item}
    Prove that $P(0)$ is true.
    Prove that Given $P(0) \ldots P(n)$ is true than $ P(n+1) $ is also true for $n \ge 0$
  \end{item}

  \begin{item}
    In the second step its being assumed that $ \frac{a^{n-1}}{a^{(n-1)-1}} $ is $a$ then in very next step everything is replaced by 1. This essentially means $a = 1$ all the time which is incorrect.
  \end{item}
  
  \begin{item}
    For $n = 1$ only the right hand side of the equation is being verified while on left hand side divide by zero operation occurs which is invalid.
  \end{item}

  \begin{item}
    for $n=1$ $F_n = 1 \ge \phi^{-1}$ (which is $0.618$)
    We may assume by induction that $F_n \ge \phi^{(n-2)}$

    Now, $F_{n+1} = F_{n-1} + F_n \ge (\phi^{n-3} + \phi^{n-2}) = \phi^{n-2}(\frac{1}{\phi} + 1) =
    \phi^{n-2} \times \phi = \phi^{n-1}$

    Hence $F_{n+1} \ge \phi^{n-1} $ for all $ n \ge 1 $
    
  \end{item}
  \begin{item}
    We may assume by induction that all the numbers till $n$ are either primes or product of primes. Now for n + 1 the number can either be prime or in form $x \times y$. As $x < (n + 1), y < (n+1)$ they themselves must be prime or product of primes. Hence $n+1$ takes the required form.
  \end{item}

  \begin{item}
    The second equation gets transformed into first one once the swap of values happens in {\it{E4}}.( $c \leftarrow d, a^\prime \leftarrow a, b^\prime \leftarrow b )$. Hence it holds true.

    The first equation is -

    $am + bn  = d $

    Substituting values this becomes

    $ (a^{\prime} - qa)m + (b^\prime - qb)n = r$

    $ a^{\prime} + b^{\prime} - q(am + bn)$

    $ c - qd =  r $ (from A6).

    $ qd - qd + r = r$

  \end{item}

  \begin{item}
    Given
    \begin{center}
      $1^2 = 1$

      $ 2^2 - 1^2 = 3 $

      $ 3^2 - 2^2 + 1^2 = 6 $

      $ 4^2 - 3^2 + 2^2 - 1^2 = 10 $

      $ 5^2 - 4^2 + 3^2 - 2^2 + 1^2 = 15 $
      \end{center}
      The formulation for this problem will be -

      \mbox{ $ n^2(-1)^{0} + (n-1)^2(-1)^1 + (n-2)^2(-1)^2 \ldots + 1(-1)^{n-1} = \frac{n(n+1)}{2}$}

      $P(1)$ is valid.

      We may assume by induction that this is valid for $n$.
      For $(n+1)^{th}$ term the series will be -

      $ {(n+1)^2(-1)^0} -1 \times {n^{th} series}  =  (n+1)^2 - \frac{n(n+1)}{2} = \frac{2(n+1)^2 - n(n+1)}{2} = \frac{(n+1) \times (n+2)}{2} $

      Hence it takes the required form and proved.
  \end{item}

  \begin{item}
    Given -
    \begin{center}
      $1^3 = 1$

      $2^3 = 3 + 5 $

      $3^3 = 7 + 9 + 11 $

      $4^3 = 13 + 15 + 17 + 19 $
 
    \end{center}
     The formulation will be -

      $n^3 = n(n-1) + 1 + n(n-1) + 3 + \ldots n(n-1) + (2n-1) $ till $n^{th}$ term
      
      Above is true for 1 hence P(1) is true.

      We may assume by induction that this is valid for $n$

      For $n+1$
      $(n+1)^3 = (n+1)(n) + 1 + (n+1)(n) + 3 \ldots (n+1)n + (2n+1)$ till $(n+1)^{th}$ term
      
      $ = (n+1)(n+1)(n) 1 + 3 + \dots + 2n+1 $
      
      $ = n(n+1)^{2} + (n+1)^2$ By {\it Eq (2)}
      
      $ = (n+1)^3$

      Hence proved.

      For b -

      $1^3 + 2^3 + 3^3 + \ldots + n^3  = (1+2+\ldots+n)^2$

      P(1) is true.

      We may assume by induction that this is valid for $n$.

      For (n+1) -

      $ 1^3 + 2^3 + 3^3 + ... + (n+1)^3 = (\frac{n(n+1)}{2})^2 + (n+1)^3 $

      $= \frac{(n+1)^2n^2}{4} + (n+1)^3 $

      $= \frac{(n+1)^2(n^2 + 4n + 4)}{4}$

      $= \frac{(n+1)^2(n+2)^2}{4}$

      $ = (\frac{(n+1)(n+2)}{2})^2 $

  \end{item}

  \begin{item}
    Inequality to prove is --- if $0 < a < 1$, then $(1-a)^n \ge 1 - na $
    
    The base case for $n = 1 $ is true as $ (1-a)^{1} \ge (1- 1 \times a) $

    We may assume by induction that this is valid for $n$

    Now for $ n + 1 $ the inequality we need to prove is -
    \begin{center}
      $(1-a)^{(n+1)} \ge (1 - (n+1)a) $

      which is $(1-a)^{n} \times (1-a) \ge 1-a - na $

      dividing both sides by $(1-a)$

      $ (1-a)^n \ge 1 - \frac{na}{1-a}$

      The above inequality is true because $ (1-a)^n \ge (1 -na) $ and $1-na \ge 1 -  \frac{na}{1-a} $ as $0 < a < 1$
      Hence the proof.
     
    \end{center}
  \end{item}

  \begin{item}
    To prove that if $n \ge 10 $, then $2^n > n^3 $

    The base case is true, as $2^{10} = 1024 > 10^3 = 1000 $
    We may assume by induction that this is valid for $n$

    For $n + 1$ the inequality will be $2^{n+1} > (n+1)^3 $

    INCOMPLETE
    
  \end{item}
  
  
\end{enumerate}
\subsubsection{Solutions for section 1.2.2}

\begin{enumerate}
  \begin{item}
    There is no smallest rational number.
  \end{item}

  \begin{item}
    $1 + 0.239999999\ldots$ is not a decimal expansion as it ends with infinite $9$s sequence.
  \end{item}

  \begin{item}
    Applying {\it eq 4} $ (-3)^{-3} = \frac{(-3)^{-2}}{-3} = \frac{(-3)^{-1}}{-3 \times -3} =
    \frac{(-3)^{0}}{-3 \times -3 \times-3} = -{\frac{1}{27}}
    $
  \end{item}

  \begin{item}
    By {\it eq 4 and 6} $(0.125)^{-\frac{2}{3}} = \sqrt[3]{0.125^{-2}} = \sqrt[3]{\frac{1}{0.125} \times \frac{1}{0.125}} = \sqrt[3]{10^6/5^6} = \frac{100}{25} = 4$
  \end{item}

  \begin{item}
    Decimal expansion -

    $n + 0.d_1d_2\ldots$

    Binary expansion - convert each digit to binary form of $0$ and $1$ once you have decimal expansion for a real number.

    $p_1p_2\ldots + 0.q_1q_2\ldots$ where $p_i, q_i \in \{0,1\}$
  \end{item}

  \begin{item}

    $x = m + 0.d_1d_2d_3\ldots$ $y = n + 0.e_1e_2e_3\ldots$

    $x = y$ if $ m = n $ and $ d_i = e_i $ for all $i$

    $x < y$ if $ m < n $ or if $ m = n $ and $d_i < e_i$ for some i such that $ d_1d_2d_3\ldots d_{i-1} = e_1e_2e_3\ldots e_{i-1} $

    $x > y$ if $ m > n $ or if $ m = n $ and $d_i > e_i$ for some i such that $ d_1d_2d_3\ldots d_{i-1} = e_1e_2e_3\ldots e_{i-1} $

    
  \end{item}

  \begin{item}
  \end{item}

  \begin{item}
  \end{item}

  \begin{item}
  \end{item}

  \begin{item}
  \end{item}

  \begin{item}
  \end{item}

  \begin{item}
  \end{item}

  \begin{item}
  \end{item}

  \begin{item}
  \end{item}

  \begin{item}
  \end{item}

  \begin{item}
  \end{item}

  \begin{item}
  \end{item}

  \begin{item}
  \end{item}

  \begin{item}
  \end{item}

  \begin{item}
  \end{item}

  \begin{item}
  \end{item}

  \begin{item}
  \end{item}

  \begin{item}
  \end{item}

  \begin{item}
  \end{item}

  \begin{item}
  \end{item}

  \begin{item}
  \end{item}

  \begin{item}
  \end{item}

  \begin{item}
  \end{item}

  \begin{item}
  \end{item}

  \begin{item}
  \end{item}


  
\end{enumerate}


\end{document}
