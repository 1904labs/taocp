\documentclass[24pt]{article}
\begin{document}

\begin{enumerate}
\begin{item}
Using a temp variable $t$ the values can be rotated like this -

$ t \leftarrow a, a \leftarrow b, b \leftarrow c, c \leftarrow d, d \leftarrow t $ 
\end{item}


\begin{item}
  At step {\bf[E3]} $r$ is assigned to $n$ and $n$ to $m$. As $r$ is reminder of division of $m$ by $n$, $r$ should be $< n$. Hence $m < n$. 
\end{item}

\begin{item}

  Below are the steps of modified algorithm {\bf[F]} which takes $m$ and $n$ as input.
  
  {\bf[F1]} Divide $m$ by $n$ and let the reminder by $r$.

  {\bf[F2]} if $r = 0$ return $n$. Terminate

  {\bf[F3]} Invoke {\bf[F]} with $n$, $r$ as input and return result.
\end{item}


\begin{item}

  57
  
\end{item}


\begin{item}

  From the procedure reading the book following properties are missing which means it's not a proper algorithm.
  \begin{itemize}
    \begin{item}
      Finiteness is missing - the whole procedure goes in a loop and does not actually terminate.
    \end{item}

    \begin{item}
      Output is missing - the procedure does not a definite output.
    \end{item}

    \begin{item}
      Effectiveness is missing - The steps cannot be done on pencil/paper or a real computer realistically.
    \end {item}

    Comparison with {\bf[E]}:
    {\bf[E]} terminates after finite number of steps and is effective(steps can be performed on pen and paper and has definite output.
  \end{itemize}
\end{item}

\begin{item}

 The answer should be close to 3. I dint calculate the exact value though.
  
\end{item}

\begin{item}

  $T_{m} +1 = U_{m}$

\end{item}


\end{enumerate}



\end{document}
