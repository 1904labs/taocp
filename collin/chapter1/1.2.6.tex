\documentclass{article}
\title{1.2.6 Binomial Coefficients}
\usepackage{mathtools}
\DeclarePairedDelimiter\ceil{\lceil}{\rceil}
\DeclarePairedDelimiter\floor{\lfloor}{\rfloor}
\begin{document}
\maketitle
\section*{Definition}
\[
  \dbinom{n}{k} = \dfrac{n!}{k!(n-k)!} = \dfrac{n(n-1)...(n-k+1)}{k(k-1)...(1)}.
\]


For example,

\[
  \dbinom{5}{3} = \dfrac{5 \cdot 4 \cdot 3}{3 \cdot 2 \cdot 1} = 10,
\]


Note how

\[
  \dfrac{n!}{(n-k)!} = n(n-1)...(n-k+1)
\]

For example, when $n=6$ and $k=3$,

\[
  {6 \cdot 5 \cdot 4 = \dfrac{6 \cdot 5 \cdot 4 \cdot 3 \cdot 2 \cdot 1}{3 \cdot
    2 \cdot 1} = \dfrac{n!}{(n-k)!}
\]

The quantity $\dbinom{n}{k}$, read \emph{''n choose k''}, is a \emph{binomial coefficient}

\section*{Operating on Binomial Coefficients}

\subsection*{A. Representation by factorials.}
  \begin{center}
    $\dbinom{n}{k} = \dfrac{n!}{k!(n-k)!}$, integer $n \leq$ integer $k \geq 0$
  \end {center}
\subsection*{B. Symmetry condition.}
\begin {center}
$\dbinom{n}{k} = \dbinom{n}{n-k}$, integer $n \leq$ integer $k$.
\end {center}

  for example:
  \begin{tabular}{ c| c c c c c c c c c c}
    n & k=0 & 1 & 2 & 3 & 4 & 5 & 6 & 7 & 8 & 9\\ 
    \hline
    8 & 1 & 8 & 28 & 56 & 70 & 56 & 28 & 8 & 1 & 0 \\  
    9 & 1 & 9 & 36 & 84 & 126 & 126 & 84 & 36 & 9 & 1    
  \end{tabular}

  notice the symetry about 4 and 4.5 for n=8 & n=9 repsectively
\subsection*{C. Moving in and out of parentheses.}
\subsection*{D. Addition formula.}
\subsection*{E. Summation formulas.}
\subsection*{F. The binomial theorem.}
\subsection*{G. Negating the upper index.}
\subsection*{H. Simplifying products.}
\subsection*{I. Sums of products.}
\end{document}